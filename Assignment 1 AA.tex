\documentclass{article}

\usepackage{amsmath}
\usepackage{amssymb}
\usepackage{graphicx}
\usepackage{setspace}

\begin{document}

\title{Home Work 0 (CS6150)}
\author{Aishwarya Asesh (u1063384)}
\maketitle
\raggedright

\textbf {Answer 1: Big Oh Notations}

(a) $\mathcal{O}(n^2)$

(b) $\mathcal{O}(\log n)$

(c) $\mathcal{O}(1/n)$

(d) $\mathcal{O}(1)$
 
(e) The sum of given series can be written in general term as $\int_{1}^{n} \frac{1}{x}dx$ which equals to $\log n$. So we can say that the complexity for this series is $\mathcal{O}(\log n)$.
\\[10pt]
\textbf {Answer 2: Removing Duplicates}

\textbf {Algorithm}: The first step will be sorting of array A using mergesort. Let array C be the sorted array. Now scan array C, if nth element is not same as (n-1)th element of array C store the value in array B.
  
\textbf {Correctness}: $100 \%$ correctness is obtaind as array C now has all the duplicate values and array B contains all distinct elements.

\textbf {Running Time}: Quicksort takes $\mathcal{O}(n\log n)$ and scanning array C for duplicates takes $\log n$. Thus the overall complexity for the whole process is $\mathcal{O}(n\log n)$.
\\[10pt]
\textbf {Answer 3: Square vs Multiply}

Let A be an algorithm that can square any n digit number in time $\mathcal{O}(\log n)$. Then the n digit square can be implied as $a.a=a^2$.
If we consider the multiplication of 2 different numbers a,b in such a way that $ab=\frac {(a+b)^2 -a^2 - b^2} {2}$, the overall time will be $\mathcal{O}{(n \log n)}$ as the process demands three calls to function A (complexity $\mathcal{O}{(n \log n)}$) and time for division ($\mathcal{O}(1) or \mathcal {O}(n)$).
\\[10pt]
\textbf {Answer 4: Probability}

\textbf {(a)}
The outcome of the toss event can be written as (H,T). If the coin is tossed k times then total possibilities of outcomes are $2^k$. The events that contain exactly one head are HTT....,THTTT.., and thus k times of them. Thus the probability is $\frac{k} {2^k}$.

\textbf{(b)}
As there are k boxes and k different colors. The total sample space of coloring each box is $k^k$.
For each box to have unique colors we have to ensure that after coloring a particular box the other box does not have the same color, that implies k,(k-1),(k-2)....1. 
Thus the probability of coloring each box uniquely is $\frac {k!}{k^k}$.
\\[10pt]
\textbf{Answer 5: Sum of Array}

\textbf{Algorithm:}
Sorting of elements in A can be done using merge sort with time complexity $\mathcal{O}(n \log n)$. Let B contain the elements of newly sorted array. For each pair of j and k, compute $A[j] + A[k]$. Check if array B contains the previous generated sum. Output YES if the result is found in array B, else Output NO.

\textbf{Correctness:}
As array B contains all the elements of array A in sorted order, for each true case of $A[i] = A[j] + A[k]$, the result must be present in array B. Thus we can expect $100\%$ correctness of algorithm.

\textbf{Running Time:}
Merge Sort takes $\mathcal{O}(n \log n)$ time complexity. Then for performing binary search for $n^2$, each takes $(n \log n)$. Thus the complexity equation is $\mathcal{O}(n^2 \log n) + \mathcal{O}(n \log n)$. Final time complexity is $\mathcal{O}(n^2 \log n)$





\end{document}
